\chapter{Calculo de Eigenvalores}
(Valores y vectores propios carecteristicos).
%--Intoducción al capítulo--%
%--Diagramas de resorte y tanque elevado--%
\begin{displaymath}
\frac{Md^2u}{dt^2}+c\frac{du}{dt}+kV=0
\end{displaymath}
\begin{displaymath}
\frac{Md^2u}{dt^2}+c\frac{du}{dt}+kV=f(t)
\end{displaymath}
\begin{displaymath}
\frac{Md^2u}{dt^2}+ku=0
\end{displaymath}
Donde $u=e^{at}$
\begin{displaymath}
\frac{du}{dt}=ae^{at} \qquad \frac{d^2V}{dt^2}=a^2e^{at}
\end{displaymath}
\begin{center}
$a^2Me^{at}+ke^{at}=0$\\
$a^2M+k=0$\\
$a^2M+k=0$
\end{center}
\begin{displaymath}
a=\sqrt{-\frac{k}{m}}=\sqrt{\frac{k}{m}}i
\end{displaymath}
\\
$Av=\lambda V$\\
$Av-\lambda V=0$\\
$(A-\lambda I)V=0\rightarrow$ Sistema lineal homogeneo tiene una soluci\'on trivial $v=0$
\\
\\
Para que $V\leq 0$
\begin{center}
$det(A-\lambda I)=0$\\ecuaci\'on polin\'omica de grado n.
\end{center} \\
\begin{displaymath}
\begin{bmatrix}
a_{11} & a_{12} \\
a_{21} & a_{22}
\end{bmatrix}-\lambda \begin{bmatrix}
1 & 0 \\
0 & 1 
\end{bmatrix}
\end{displaymath}
\\
\begin{displaymath}
det\left| \begin{matrix}
a_{11}-\lambda & a_{12} \\
a_{21} & a_{22}-\lambda
\end{matrix}\right| =0
\end{displaymath}


 