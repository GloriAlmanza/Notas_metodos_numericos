\chapter{Calculo de Eigenvalores}
(Valores y vectores propios carecteristicos).
%--Intoducción al capítulo--%
%--Diagramas de resorte y tanque elevado--%
\begin{displaymath}
\frac{Md^2u}{dt^2}+c\frac{du}{dt}+kV=0
\end{displaymath}
\begin{displaymath}
\frac{Md^2u}{dt^2}+c\frac{du}{dt}+kV=f(t)
\end{displaymath}
\begin{displaymath}
\frac{Md^2u}{dt^2}+ku=0
\end{displaymath}
Donde $u=e^{at}$
\begin{displaymath}
\frac{du}{dt}=ae^{at} \qquad \frac{d^2V}{dt^2}=a^2e^{at}
\end{displaymath}
\begin{center}
$a^2Me^{at}+ke^{at}=0$\\
$a^2M+k=0$\\
$a^2M+k=0$
\end{center}
\begin{displaymath}
a=\sqrt{-\frac{k}{m}}=\sqrt{\frac{k}{m}}i
\end{displaymath}
\\
$Av=\lambda V$\\
$Av-\lambda V=0$\\
$(A-\lambda I)V=0\rightarrow$ Sistema lineal homogeneo tiene una soluci\'on trivial $v=0$
\\
\\
Para que $V\leq 0$
\begin{center}
$det(A-\lambda I)=0$\\ecuaci\'on polin\'omica de grado n.
\end{center} 

\begin{displaymath}
\begin{bmatrix}
a_{11} & a_{12} \\
a_{21} & a_{22}
\end{bmatrix}-\lambda \begin{bmatrix}
1 & 0 \\
0 & 1 
\end{bmatrix}
\end{displaymath}
\\
\begin{displaymath}
det\left| \begin{matrix}
a_{11}-\lambda & a_{12} \\
a_{21} & a_{22}-\lambda
\end{matrix}\right| =0
\end{displaymath}
\\
\begin{center}
$(a_{11}-\lambda)(a_{22}-\lambda)-a_{21}a_{12}=0$\\
$\lambda ^2-(a_{11}+a_{22})\lambda +a_{11}a_{22}-a_{11}a_{22}-a_{11}a_{12}=0$\\
\end{center}
\begin{displaymath}
\lambda =\frac{a_{11}+a_{22}\pm\sqrt{(a_{11}+a_{22})^2-4(a_{11}a_{22}-a_{21}a_{12})}}{2}
\end{displaymath}
\begin{displaymath}
-\frac{a_{21}}{a_{11}\lambda}\begin{bmatrix}
a_{11}-\lambda_1 & a_{12} \\
a_{21} & a_{22}\lambda_2
\end{bmatrix}\begin{bmatrix}
x\\y
\end{bmatrix}=\begin{bmatrix}
0 \\ 0
\end{bmatrix}
\end{displaymath}

\begin{displaymath}\begin{bmatrix}
a_{11}-\lambda_1 & a_{12} \\
0 & 0
\end{bmatrix}\begin{bmatrix}
x\\y
\end{bmatrix}=\begin{bmatrix}
0 \\ 0
\end{bmatrix}
\end{displaymath}
\begin{displaymath}
v_1=\begin{bmatrix}
-\frac{a_{12}y}{a_{11}-\lambda_1} \\
y
\end{bmatrix} \qquad v_2=\begin{bmatrix}
-\frac{a_{12}y}{a_{11}-\lambda_2} \\
y
\end{bmatrix}
\end{displaymath}
\subsection*{Propiedades}
\begin{itemize}
\item Los eigenvectores son linealmente independientes
\item Los igenvectores no son \'unicos, pero sus direcciones si lo son.
\item Si a es sim\'etrico, todos los eigenvectores son reales.
\item El rango de una matriz esta relacionado con el n\'umero de eigenvectores nulos
\end{itemize}
\section{M\'etodo de la potencia}
\subsection*{Hipotesis fundamental:}
Los eigenvalores de $A$, se pueden ordenar por valor absoluto en la forma
\begin{displaymath}
|\lambda_1|>|\lambda_2|\geq |\lambda_3|\geq|\lambda_4|\geq\cdots\geq|\lambda_n|
\end{displaymath}

Entonces empezando en un vector cualquiera $z_o\leq 0$ se tiene que la secuencia $z_{k+1}=Az_k$ converge a la direcci\'on $v_1$ si $k \to \infty$\\
\begin{displaymath}
Calcular \quad v_1=\frac{z_k}{\parallel z_k \parallel}
\end{displaymath}
\begin{center}
$v_1^TAv_1=r_1^T\lambda_1v_1$\\
$v_1^TAV_1=\lambda_1v_1^TV_1$\\
$\lambda_1=v_1^TAV_1$
\end{center}
Normalizar el vector resultante para obtener resultados aceptables y no haya problema de desbordamientp.
%---Algoritmo de potencia---%

\section{M\'etodo de iteraci\'on inversa}
Es un m\'etodo \'util cuando la matriz es simetrica y definida positiva.
\subsection*{Teorema}
Si $det(A)\leq 0,\quad A^{-1}$ existe y tiene los misos eigenvectors, corresponde a los de A por 
\begin{displaymath}
\frac{1}{\lambda_1},\frac{1}{\lambda_2},\frac{1}{\lambda_3},\cdots,\frac{1}{\lambda_n},
\end{displaymath}
%---Algoritmo de iteración inversa---%

\section{T\'ecnica de desplazamiento ("Shifting")}
Cuando se quieren encontrar los eigenvectores m\'as chiquitos, siempre y cuando la matriz sea definida positiva.\\
sirve para acelerar la convergencia, con el m\'etodo de la potencia trasladada.\\ 
$Av=\lambda v$\\
La matriz $A+\theta I$ tiene los mismos eigenvectores que $A$, y sus eigenvectores son $\theta +\lambda$.
$(A+\theta I)v=Av+\theta Iv=Av+\theta v$
\\
$(A+\theta I)v=Av+\theta Iv=\lambda v+\theta v$\\
$(A+\theta I)v=(\lambda+\theta)v=(\lambda+\theta)v$
\section{T\'ecnicas de deflaci\'on}
Se obtiene una nueva matriz $B$ de manera que contenga los mismos eigenvectores de $A$, pero uno de sus eigenvectores se reemplaza por uncero.
\begin{center}
$Bv=\lambda 'v$\\
$Av=\lambda v$
\end{center}
Si $A$ es sim\'etrica, $B$ se obtiene por $B=A-\lambda_1v_1v_1^T$
 
\section{M\'etodo del polinomio}
se puede aplicar la t\'ecnica de desplazamiento.
\begin{displaymath}
det(A-I\lambda)=0
\end{displaymath} 
\begin{center}
$f(\lambda)=0$\\
$f(x)=0$\\
$LU=A$\\
$det(A)=det(L)det(U)$
\end{center}