\chapter{Interpolaci\'on}
%Imagen de grafica%
%Imagen de grafica%
%imagen de grafica%

%insertar ecuaciones%
\{x;y;\} y=f(x)
$E=\frac{1}{2}\sum(yi-f(xi))$%formula%
f(x) es lineal en los par\'ametros\\
%formula%
aj son los par\'ametros de interpolaci\'on
yi(x) es una funcion exclusivamente de x

%formulas%

%matriz%

\section{M\'inimos cuandrados no lineales}
%formulas%

\section{M\'etodo de Gauss-Newton}
%formulas%

\section{Interpolaci\'on polin\'omica}
\subsection{Interpolaci\'on por segmentarias}
%formulas%
%grafica de interpolacion%
\begin{itemize}
\item Pedazos de polinomios
\item Que salgan curvas suavecitas
\item Son dependientes de la orintaci\'on del sistema de coordenadas

\item Si la continuidad s\'olo se exige en valores, clase CO
\item Cuando la continuidad es en pendientes clase C1

\item Si se exige continuidad en pendiente y curvatura, clase C2 (evaluar en segundas derivadas)

Caracter\'isticas\\
1.Grado del polinomio\\
2.Clase de la segmentaria. (Tipo de continuidad en los tipos de uni\'on)
\end{itemize}

%Grafica segmentarias cubicas de clase C1%
Cada tramo tiene 4 inc\'ognitas.\\
Por cada punto hay una ecuaci\'on.\\
M\'as una adicional que se deriva (ecuaci\'on de enlace).

\section{Integraci\'on num\'erica}

Teorema fundamental del c\'alculo integral
%Formula%

Teorema de Green 2D (Simplificar integrales)
%Formulas%

Teorema de divergencia de Gauss
Teorema de Stokes

%Formulas%

Formulas de Newton - Cotes
Cuadraturas de Gauss
%Formulas%
%Grafica%
%Formulas%
%Grafica%
%Formulas%
%Grafica%

\subsection{Por trapecio}
%formulas

\subsection{Sympson 3 puntos}
%formulas%

\subsection{Para 4 puntos}
%formulas%

\section{Cuadraturas Gaussianas}
Se escoge la posici\'on \'optima de los puntos de integraci\'on para poder asegurar el resultado

%Formulas%
%Graficas%
%Graficas%

\subsection{Cuadratura de orden 1}
%Formulas%

\subsection{Cuadratura de orden 2}
%Formulas%

\subsection{Cuadratura de orden 3}
%Formulas%

\subsection{Cuadraturas Gaussianas}
Da la soluci\'on de polinomis P = 2n+1
%Formulas%
%Graficas%
%Graficas%


\subsection{Cuadratura de orden 1}
%Formulas%

\subsection{Cuadratura de orden 2}
%Formulas%

\subsection{Cuadratura de orden 3}
%Formulas%

\subsection{Cuadraturas Gaussianas}
Da la soluci\'on de polinomios P = 2n+1

\section{Integrales M\'ultiples}

\section{F\'ormulas de Newton - Cotes}

\section{Indeterminaciones}

\section{Ecuaciones Diferenciales}

\section{Soluci\'on de sistemas no lineales}

\section{Eigenvalores y eigenvectores}
Los vectores propios son los vectores no nulos que cuando son transformados por el operador dan lugar a un m\'ultiplo escalar de s\'i mismos, con lo que no cambian su direcci\'on, este escalar se llama eigenvalor. Los eigenvalores son las ra\'ices del polinomio caracter\'istico.

\begin{itemize}
\item {M\'etodo de la potencia}\\
\noindent Este m\'etodo puede encontrar el eigenvalor m\'as grande en valor absoluto y su correspondiente eigenvector. No precisan del c\'alculo polin\'omio caracter\'istico

\item {M\'etodo de la potencia inversa}\\
\noindent C\'alcula el valor propio de m\'odulo m\'inimo y un vector propio asociado. La matriz A debe ser invertible.

\item {M\'etodo de la potencia inversa desplazada}
\noindent Aproximar un valor propio $\lambda$ y un vector propio asociado a partir de la estimaci\'on %poner lambda con la linea arriba%  
$\simeq \lambda$

\item {M\'etodo del polinmio}
\noindent Sensible al desbordamiento, se puede aplicar desplazamiento.
\end{itemize}

%La siguiente hoja creo que esta desfasada, pero estoy transcribiendo en orden cronologico, necesito preguntar bien el orden%

Interpolaci\'on: Proceso en el cual se calculan valores num\'ericos desconocidos a partir de otros ya conocidos mediante la aplicaci\'on de algoritmos concretos.\\
Encontrar un valor intermedio entre dos o m\'as puntos base conocidos, los cuales se pueden aproximar mediante polinomios.\\

Interpoaci\'on lineal: Se interpola con l\'ineas rectas entre una serie de puntos ($x_0$, $y_0$). La idea b\'asica es conectar los 2 puntos dados en $x_i$, es decir ($x_0$,$y_0$) y %checar esta parte% 
($x_1$,$y_1$). 
La funci\'on interpolante es una l\'inea recta entre los dos puntos. La funci\'on se sustituye por la recta que pasa por los puntos conocidos.\\

M\'inimos cuadrados lineales\\
M\'inimos cuadrados no lineales\\
M\'etodo de Gauss Newton: Para funciones no lineales (logaritmos, exponenciales y trigonom\'etricas)\\

Interpolaci\'on polin\'omica: La funci\'on inc\'ognita se sustituye por un polinomio que coincide con aquella en los puntos conocidos.\\

Polinomios de Lagrange: Encontrar una funci\'on polin\'omica que pase por esos n+1 puntos y que tengan el menor grado posible un polinomio que pase por varios puntos determinados se llama polinomio de interpolaci\'on. El polinomio de Lagrange pasan por todos los n+1 puntos dados:\\
%Poner grafica%

Interpolaci\'on por segmentarias o splines: El t\'ermino spline referencia a una amplia clase de funciones que son utilizadas en aplicaciones que requieren la interpolaci\'on de datos o un suavizado.\\ %checar otro apunte, ultimas 3-4 palabras, no estoy segura de la redaccion% Su construcci\'on consiste en obtener una funci\'on de interpolaci\'on para cada segmento.
El spline c\'ubico (k=3) es el spline m\'as empleada, debido a que proporciona un ajuste a los puntos tabulados y su c\'alculo no es excesivamente complejo.\\