\chapter{Interpolaci\'on}
%Imagen de grafica%
%Imagen de grafica%
%imagen de grafica%

%insertar ecuaciones%
\{x;y;\} y=f(x)
$E=\frac{1}{2}\sum(yi-f(xi))$%formula%
f(x) es lineal en los par\'ametros\\
%formula%
aj son los par\'ametros de interpolaci\'on
yi(x) es una funcion exclusivamente de x

%formulas%

%matriz%

\section{M\'inimos cuandrados no lineales}
%formulas%

\section{M\'etodo de Gauss-Newton}
%formulas%

\section{Interpolaci\'on polin\'omica}
\subsection{Interpolaci\'on por segmentarias}
%formulas%
%grafica de interpolacion%
\begin{itemize}
\item Pedazos de polinomios
\item Que salgan curvas suavecitas
\item Son dependientes de la orintaci\'on del sistema de coordenadas

\item Si la continuidad s\'olo se exige en valores, clase CO
\item Cuando la continuidad es en pendientes clase C1

\item Si se exige continuidad en pendiente y curvatura, clase C2 (evaluar en segundas derivadas)

Caracter\'isticas\\
1.Grado del polinomio\\
2.Clase de la segmentaria. (Tipo de continuidad en los tipos de uni\'on)
\end{itemize}

%Grafica segmentarias cubicas de clase C1%
Cada tramo tiene 4 inc\'ognitas.\\
Por cada punto hay una ecuaci\'on.\\
M\'as una adicional que se deriva (ecuaci\'on de enlace).

\section{Integraci\'on num\'erica}

Teorema fundamental del c\'alculo integral
%Formula%

Teorema de Green 2D (Simplificar integrales)
%Formulas%

Teorema de divergencia de Gauss
Teorema de Stokes

%Formulas%

Formulas de Newton - Cotes
Cuadraturas de Gauss
%Formulas%
%Grafica%
%Formulas%
%Grafica%
%Formulas%
%Grafica%

\subsection{Por trapecio}
%formulas

\subsection{Sympson 3 puntos}
%formulas%

\subsection{Para 4 puntos}
%formulas%

\section{Cuadraturas Gaussianas}
Se escoge la posici\'on \'optima de los puntos de integraci\'on para poder asegurar el resultado

%Formulas%
%Graficas%
%Graficas%

\subsection{Cuadratura de orden 1}
%Formulas%

\subsection{Cuadratura de orden 2}
%Formulas%

\subsection{Cuadratura de orden 3}
%Formulas%

\subsection{Cuadraturas Gaussianas}
Da la soluci\'on de polinomis P = 2n+1