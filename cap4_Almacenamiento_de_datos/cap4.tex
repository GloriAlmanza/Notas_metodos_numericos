\chapter{Almacenamiento de datos }
El almacenamiento de datos es un tema muy importante dentro de los m\'etodos num\'ericos, ya que es necesario tener un computador capaz de almacenar y resolver r\'apidamente los sistemas.

Alguanas ocaciones los sistemas de soluci\'on son muy grandes lo que implica un gran espacio de almancenamiento y esto puede provocar que la velocidad de soluci\'on no sea tan buena. Es por eso que se implementan t\'ecnicas de reducci\'on matricial, para acelerar el tiempo de soluci\'on de sistemas robustos sin afectar el resultado.
\subsection{Ancho de banda}

\subsection{Paralelizar}

\section{Almacenamiento Reducido}
Antes es importante recordar el tamaño de algunos tipos de datos y las equivalencias existentes entre tamaños de datos. Para los datos tipo int recordemos necesita un espacio de almacenamiento de 4 bytes y para los tipo double 8 bytes (son los tipos mas usados).En cuanto a las equivalencias, en seguida se muentran en la tabla.
\begin{tabular}{| c | c |}
%---Completar---%
\end{tabular}
\subsection{Matriz sin reducci\'on o con ancho de banda completa}
Para calcular el espacio que ocupa una matriz de tamaño $m \times n$:
\begin{displaymath}
n\times m\times (cantidad\quad bytes*)
\end{displaymath}
*(depende el tipo de dato) 


 