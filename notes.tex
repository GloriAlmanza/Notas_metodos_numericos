%%%------Preambulo------%%%

%Tipo de documento:
\documentclass[a4paper]{article} 
%---------------------------------------------%
\usepackage[spanish,mexico]{babel}
\usepackage[utf8]{inputenc}
\usepackage{amssymb}
\usepackage{graphicx}
\usepackage{verbatim}
\usepackage{amsmath}
\usepackage{multicol}

\usepackage{tcolorbox}
\tcbuselibrary{skins}

%Titulo:
\title{
\vspace{-3em}
\begin{tcolorbox}[colframe=white,opacityback=0]
\begin{tcolorbox}
\Huge\sffamily Notas de Métodos Numéricos   
\end{tcolorbox}
\end{tcolorbox}
\vspace{-3em}
}
%---------------------------------------------%

\date{}

\usepackage{background}
\SetBgScale{1}
\SetBgAngle{0}
\SetBgColor{red}
\SetBgContents{\rule[0em]{4pt}{\textheight}}
\SetBgHshift{-2.3cm}
\SetBgVshift{0cm}

\usepackage{lipsum}% just to generate filler text for the example
\usepackage[margin=2cm]{geometry}

\usepackage{tikz}
\usepackage{tikzpagenodes}

\parindent=0pt

%Formato de topic 
\usepackage{xparse}
\DeclareDocumentCommand\topic{ m m g g g g g}
{
\begin{tcolorbox}[sidebyside,sidebyside align=top,opacityframe=0,opacityback=0,opacitybacktitle=0, opacitytext=1,lefthand width=.3\textwidth]
\begin{tcolorbox}[colback=red!05, colframe=red!25, sidebyside align=top, width=\textwidth, before skip=0pt]
#1.\end{tcolorbox}%
\tcblower
\begin{tcolorbox}[colback=blue!05,colframe=blue!10,width=\textwidth,before skip=0pt]
#2
\end{tcolorbox}
\IfNoValueF {#3}{
\begin{tcolorbox}[colback=blue!05,colframe=blue!10,width=\textwidth]
#3
\end{tcolorbox}
}
\IfNoValueF {#4}{
\begin{tcolorbox}[colback=blue!05,colframe=blue!10,width=\textwidth]
#4
\end{tcolorbox}
}
\IfNoValueF {#5}{
\begin{tcolorbox}[colback=blue!05,colframe=blue!10,width=\textwidth]
#5
\end{tcolorbox}
}
\IfNoValueF {#6}{
\begin{tcolorbox}[colback=blue!05,colframe=blue!10,width=\textwidth]
#6
\end{tcolorbox}
}
\IfNoValueF {#7}{
\begin{tcolorbox}[colback=blue!05,colframe=blue!10,width=\textwidth]
#7
\end{tcolorbox}
}
\end{tcolorbox}
}
%---------------------------------------------%

%Formato de resumen/Comentarios 
\def\summary#1{
\begin{tikzpicture}[overlay,remember picture,inner sep=0pt, outer sep=0pt]
\node[anchor=south,yshift=-1ex] at (current page text area.south) {% 
\begin{minipage}{\textwidth}%%%%
\begin{tcolorbox}[colframe=white,opacityback=0]
\begin{tcolorbox}[enhanced,colframe=black,fonttitle=\large\bfseries\sffamily,sidebyside=true, nobeforeafter,before=\vfil,after=\vfil,colupper=black,sidebyside align=top, lefthand width=.95\textwidth,opacitybacktitle=1, opacitytext=1,
segmentation style={black!55,solid,opacity=0,line width=3pt},
title=Resumen / Comentarios
]
#1
\end{tcolorbox}
\end{tcolorbox}
\end{minipage}
};
\end{tikzpicture}
}
%---------------------------------------------%

%Comienzan Notas
\begin{document} 
\maketitle

%% INICIA TOPIC -------------------------------------------------------------------------------- %%
\topic{Anotaciones / Preguntas}%
{\section{Conceptos importantes}}{Computadora : Máquina electronica de cálculo, compuesta por circuitos lógicos que generan conexiones.
Componentes de circuitos lógicos:
\begin{itemize}
\item Amplificador operacional:
\item Biestable:
\item PDL:
\item Diac:
\item Diodo:
\item FGPA: 
\item Memoria:
\item Microprocesador:
\item Pila:
\item Tiristor:
\item Puerta lógica:
\item Transistor:
\item Triac: 
\end{itemize}
%
}{%----Faltan imagenes de los elementos----%
%----Tabla de almacenamiento de datos-----%
Representaci\'on de punto flotante \\
%----Ilustración de celdas-----%
$a \times 10^b$\\
$1 \textgreater |a| \geq 0.1$\\
exceptuando cuando $a=0.0b$ \\
\begin{tabular}{| c | c |}
\hline
Tipo de datos & espacio de almacenamiento\\
\hline 
float& 4 bytes\\
double & 8 bytes\\
long double & 16 bytes\\
\hline
\end{tabular}
\\ \\
Error de corte:
\\ \\
Epsil\'on de la m\'aquina (EPS): El EPS es el n\'umero m\'as pequeño tal que $(1+EPS)\textgreater1$ para la m\'aquina que realiza la suma.
}{El siguiente c\'odigo permite conocer el epsil\'on de tu computadora.}
%----Código para el epsilón----%
%% FINALIZA TOPIC -------------------------------------------------------------------------------- %%

\newpage
%% INICIA TOPIC -------------------------------------------------------------------------------- %%
\topic{Anotaciones / Preguntas}
{Tipos de Errores}{
Error de redondeo: P\'erdida de cifras decimales a medida que se aumenta el exponente.
%---Ilustración de recta numérica---%
\\ \\
Error de truncamiento \\
Teorema de Taylor: Si $f(x)$ es una funci\'on suave en un intervalo abierto $(a,b)$ que contiene a $c$, para un n\'umero $c+h$ contenido en $(a,b)$ \\
$f(c+h)=f(c)+f'(c)h+f''(c)\frac{h^3}{3!}+f'''(c)\frac{h^3}{3!}+...+f^n(c)\frac{h^n}{n!}$ \\
El hecho de perder cifras debido a limitar el resultado a ciertas decimale es a lo que llamamos error de truncamiento.
Complejidad algor\'itmica y costo computacional:
%---Diagramas---%
\\ \\ 
Tiempo de tendencia a funciones\\
\begin{tabular}{| c | c |}
\hline
$log(n)$ & \\
$n$ & Tiempos lineales \\ 
$nlog(n)$ & \\
\hline
$n^2$ & \\
$n^3$ & Tiempos polin\'omicos \\ 
$n^4$ & \\
\hline
$2^n$ & NP-Duro \\
$n!$ & NP-Cmpleto\\
\hline
\end{tabular}
%-Ilustracioes de graficas-%
\\ \\
Convergencia\\
Un ciclo de c\'alculo se traduce a una iteraci\'on 
%--Ilustración de iteración--%
\\
Sea $x_k$ una sucesión de valores. Si existe un n\'umero $x^*$ tal que\\
$\lim\limits_{k\to\infty}x_k=x^*$ \\
La sucesi\'on converge a $x^*$, si eso no ocurre, entonces la suseci\'on diverge.
\\ \\ 
Velocidad de convergencia\\ 
${x_k}$ converge s $x^*$ \\
\begin{enumerate}
\item Si existe un $k\geq 1$ a partir del cual se observa que $|x_{k+1}-x^*|\leq C|x_k-x^*|$ donde $C$ es constante entre $(0,1)$ se tiene velocidad de convergencia lineal.
\item Igual que el anterior pero $|x_{k+1}-x^*|\leq c_k|x_k-x^*|$ con $c_k\exists(0,1)$ y $C_k\to0$ cuando $k\to\infty$, se tiene  velocidad de convergencia lineal.
\item A partir de la iteraci\'on $k$ se observa que\\
$|x_{k+1}-x^*|\leq C{|x_k-x^*}^P$ donde $C$y $P$ son constantes $C\exists(0,1)$ y $P\geq2$, se tiene convergencia de orden $P$
\end{enumerate}
}


%Línea de comentarios
\summary{\vspace{2.5cm}.}

\end{document}