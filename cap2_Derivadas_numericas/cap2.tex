\chapter{Derivadas n\'umericas}
%-----Concepto básico de derivada númerica y utilidad------%
\begin{center}
\begin{eqnarray}
\nonumber
\frac{d(f(x))}{dx}=\lim\limits_{\Delta x\to0}\frac{f(x+\Delta x)-f(x)}{\Delta x} \qquad\nonumber h\geq\sqrt{EPS}
\end{eqnarray}
\end{center}
\begin{tabular}{ l }
 Derivada n\'umerica hacia adelante \\
$f'(x)\thickapprox \frac{f(x+h)-f(x)}{h}$ \\
\\
 Derivada n\'umerica hacia atr\'as \\
$f'(x)\thickapprox \frac{f(x)-f(x-h)}{h}$ \\
\\
 Derivada n\'umerica al centro \\
$f'(x)\thickapprox \frac{f(x+h)-f(x-h)}{2h}$ \\
\\
\end{tabular}
\\
Para el c\'alulo de derivadas n\'umericas se parte del analis\'is de una segunda deribada hacia adelante.\\
\begin{displaymath}
f''(x)=\frac{f'(x+h)-f'(x)}{h}=\frac{\frac{f(x+2h)-f(x+h)}{h}-\frac{f(x+h)-f(x)}{h}}{h}
\end{displaymath} 
\\De la cual obtenemos la ecuaci\'on para calcular $f''$ hacia adelante\\ \\
\begin{tabular}{  l  }
$f''(x)=\frac{f(x+2h)-2f(x+h)+f(x)}{h^2}$ \\
Segunda derivada n\'umerica hacia adelante \\
\\
$f''(x)=\frac{f(x)-2f(x-h)+f(x-2h)}{h^2}$ \\
Segunda derivada n\'umerica hacia atr\'as \\
\\
$f''(x)=\frac{f(x+h)-2f(x)+f(x-h)}{h^2}$ \\
Segunda derivada n\'umerica al centro \\
\\
\end{tabular}
\\
Para obtener un buen resultado en en el c\'alculo de segundas derivadas num\'ericas es necesario que $h\geq\sqrt[3]{EPS}$ \\ \\ \\
Ejemplo\\ \\
C\'alcular la $1^{ra}$ y $2^{da}$ derivada en $x=0.78$ de la funci\'on $3x^2\sin^2x$\\ \\
Anal\'iticamente \\
\begin{displaymath}
u=3x^2\quad v=\sin^2x
\end{displaymath}
